\chapter{Einleitung}

In den folgenden Laborversuchen werden die wichtigsten Grundlagen der Elektrotechnik praktisch untersucht. Der Bericht ist in zwei Hauptteile unterteilt: Gleichstrom und Wechselstrom.


\section{Gleichstromtechnik}
Im ersten Teil geht es darum, wie sich Strom und Spannung in einfachen Stromkreisen verhalten. Folgende Punkte werden untersucht:
\begin{itemize}[itemsep=0pt]
\item Das ohmsche Gesetz
\item Verteilung von Strom und Spannung in Reihenschaltungen und Parallelschaltungen
\item Verhalten von Spannungsquellen in verschiedensten Schaltungen
\end{itemize}

\section{Wechselstromtechnik}
Der zweite Teil behandelt die Versuche zur Wechselstromtechnik. Folgende Punkte werden untersucht:
\begin{itemize}[itemsep=0pt]
	\item Die Reaktion von Widerständen, Spulen oder Kondensatoren auf Wechselspannung 
	\item Untersuchung der zeitlichen Verschiebung zwischen Strom und Spannung
	\item Berechnung der Leistungen
\end{itemize}
